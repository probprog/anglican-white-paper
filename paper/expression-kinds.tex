\subsection{Expression Kinds}
\label{sec:expr}

There are three different kinds of inputs to CPS transformation:
\begin{itemize}
    \item Literals, which are constant expressions. They
       are passed as an argument to the
        continuation unmodified.
    \item Value expressions such as the \texttt{fn} form
        (called \textit{opaque}
        expressions in the code). They must be transformed to
        CPS, but the transformed object is passed to the
        continuation as a whole, opaquely.
    \item General expressions (which we call
        \textit{transparent} expressions).
        The continuation is threaded through such an expression
        in an expression-specific way, and can be
        called in multiple locations of the CPS-transformed
        code, such as in all branches of an \texttt{if}
        statement.
\end{itemize}

\subsubsection{Literals}
\label{sec:literals}

Literals are the same in Anglican and Clojure. They are
left unmodified; literals are a subset of opaque expressions.
However, the Clojure syntax has a peculiarity
of using the syntax of compound literals (vectors, hash maps,
and sets) for data constructors. Hence, compound literals must
be traversed recursively, and if there is a nested non-literal
component, transformed into a call to the corresponding data
constructor. Functions \texttt{cps-of-vector},
\texttt{cps-of-hash-map}, \texttt{cps-of-set}, called from
\texttt{cps-of-expression}, transform Clojure constructor syntax
(\texttt{[...]}, \texttt{\{...\}}, \texttt{\#\{...\}}) into the
corresponding calls:
\begin{lstlisting}[style=default]
=> (cps-of-vector [0 1 2] 'cont)
(cont (vector 0 1 2) $state)
=> (cps-of-hash-map {:a 1, :b 2} 'cont)
(cont (hash-map :a 1, :b 2) $state)
=> (cps-of-set #{0 1} 'cont)
(cont (set (list 0 1)) $state)
\end{lstlisting}

\subsubsection{Opaque Expressions}
\label{sec:opaque}

Opaque, or value, expressions, have a different shape in the
original and the CPS form. However, their CPS form follows the
pattern \texttt{(continuation transformed-expression \$state)}, and thus
the transformation does not depend on the continuation parameter, and
can be accomplished without passing the parameter as a
transformation argument. Primitive (non-CPS) procedures used in
Anglican code, \texttt{(fn ...)} forms, and \text{(mem ...)}
forms are opaque and transformed by
\texttt{primitive-}\linebreak[0]\texttt{procedure-}\linebreak[0]\texttt{cps},
\texttt{fn-cps}, and \texttt{mem-cps}, correspondingly: a
slightly simplified CPS form of expression
\begin{lstlisting}[style=default]
(fn [x y]
  (+ x y))
\end{lstlisting}
would be
\begin{lstlisting}[style=default]
(fn [cont $state x y]
  (cont (+ x y) $state))
\end{lstlisting}
In the actual code an automatically generated fresh symbol is
used instead of \texttt{cont}.

\subsubsection{General Expressions}
\label{seq:general}

The most general form of CPS transformation receives an
expression and a continuation as parameters, and returns the expression
in CPS form with the continuation parameter potentially called in multiple
tail positions. General expressions can be somewhat voluntarily
divided into several groups:
\begin{itemize}
    \item binding forms --- \texttt{let} and
        \texttt{loop/recur};
    \item flow control --- \texttt{if}, \texttt{when},
        \texttt{cond}, \texttt{case}, \texttt{and}, \texttt{or} and \texttt{do};
    \item function applications and \texttt{apply};
    \item probabilistic forms --- \texttt{observe},
        \texttt{sample}, \texttt{store}, and \texttt{retrieve}.
\end{itemize}
Functions that transform general expressions accept the
expression and the continuation as parameters, and are
consistently named \texttt{cps-of-}\textit{form}, for example,
\texttt{cps-of-do}, \texttt{cps-of-store}.
