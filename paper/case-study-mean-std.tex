\begin{lstlisting}[style=default]
;; single customer                      
(def time-to-arrive+ 
    (map (fn [x] (first (:times-to-arrive x)))
         (filter :same-customer results)))
(def mean-to-arrive+ (mean time-to-arrive+))
(def sd-to-arrive+ (std time-to-arrive+))

;; two customers
(def times-to-arrive+ 
  (map :times-to-arrive 
    (filter (fn [x] (not (:same-customer x))) results)))
(def mean-1-to-arrive+ (mean (map first times-to-arrive+)))
(def sd-1-to-arrive+ (std (map first times-to-arrive+)))
(def mean-2-to-arrive+ (mean (map second times-to-arrive+)))
(def sd-2-to-arrive+ (std (map second times-to-arrive+)))
\end{lstlisting}

In addition to plotting the distribution histograms, we use
functions \texttt{mean}, \texttt{std} provided
in the \texttt{anglican.{\linebreak[0]}stat} namespace
along with other useful statistical functions to compute
summary statistics. This is another illustration of advantage of
tight integration between Clojure and Anglican --- probabilistic
models are expressed in Anglican. However, processing of data
and results can rely on the full power of Clojure.
