For inference, we chose to use  Lightweight
Metropolis-Hastings, perhaps somewhat arbitrarily. A strength of
probabilistic programming is that models are separated from
inference. To switch to a different inference algorithm we just
need to pass a different value to \texttt{doquery}. For example,
we may decide to use Black-Box Variational Bayes (BBVB), which
may not work equally well for all probabilistic programs, but is
much faster to converge.

\begin{lstlisting}[style=default]
(def samples (doquery :bbvb deli nil))
\end{lstlisting}

We can still use samples and summary statistics with BBVB to
approximate the posterior distribution. However, variational
inference approximates the posterior by known distributions,
and we can directly retrieve the distribution parameters.

\begin{lstlisting}[style=default]
(clojure.pprint/pprint
  (anglican.bbvb/get-variational 
    (nth samples N)))
\end{lstlisting}

\begin{lstlisting}[style=default]
{S28209
 {(0 anglican.runtime.normal-distribution)
  {:mean 10.99753360180663,
   :sd 0.7290976433082352}},
 S28217
 {(0 anglican.runtime.normal-distribution)
  {:mean 12.668050292254202,
   :sd 0.9446695174790353}},
 S28215
 {(0 anglican.runtime.normal-distribution)
  {:mean 9.104132559955836,
   :sd 0.9479290526821788}},
 S28219
 {(0 anglican.runtime.flip-distribution)
  {:p 0.11671977016875924,
   :dist {:min 0.0, :max 1.0}}}}
\end{lstlisting}

We can guess that the variational distributions correspond to the
prior distributions in the \texttt{sample} forms, in the order
of appearance. However, it would help if we could use more
informative labels instead of automatically generated symbols
\texttt{S28209}, \texttt{S28217}, etc. Here the option to
specify identifiers for probabilistic forms explicitly comes
handy. If we modify the \texttt{sample} forms to use explicit identifiers
(only the forms are shown for brevity), the output becomes much
easier to analyse.

\begin{lstlisting}[style=default]
(defm same-customer 
      ...
      (sample :arrival-time-same
              time-to-arrive-prior)]
      ...)

(defm different-customers
      ... 
      (sample :arrival-time-first
              time-to-arrive-prior)
      ...
      (sample :arrival-time-second
              time-to-arrive-prior)]
      ...)

(defquery deli ...
      ...
      (sample :same-or-different
              (flip p-same))
      ...)
\end{lstlisting}

The output becomes much easier to interpret and analyse
programmatically.

\begin{lstlisting}[style=default]
{:arrivate-time-same
 {(0 anglican.runtime.normal-distribution)
  {:mean 10.99753360180663,
   :sd 0.7290976433082352}},
 :arrival-time-first
 {(0 anglican.runtime.normal-distribution)
  {:mean 12.668050292254202,
   :sd 0.9446695174790353}},
 :arrival-time-second
 {(0 anglican.runtime.normal-distribution)
  {:mean 9.104132559955836,
   :sd 0.9479290526821788}},
 :same-or-different
 {(0 anglican.runtime.flip-distribution)
  {:p 0.11671977016875924,
   :dist {:min 0.0, :max 1.0}}}}
\end{lstlisting}

